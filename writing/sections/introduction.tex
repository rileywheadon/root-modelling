Root growth in \emph{Arabidopsis Thaliana} is a tightly regulated process governed by a network of interacting hormones. Central to this network is the plant hormone auxin, which has been linked to nearly every aspect of cell behaviour within the root \cite{saini2013}. Additionally, at least eight other substances \cite{tian2018}, including cytokinin, brassinosteroids (BR), and ethylene have been shown to contribute to promoting robust and adaptable hormone gradients within the root. The extensive measurement of these hormones, their derivatives, and behaviour \cite{verbelen2006, kramer2007, marhavy2011, ambrose2013} has provided fertile ground for mathematical modelling. While various approaches have been employed in modelling these hormone gradients \cite{rutten2022}, this paper will explore two dimensional models on a vertical cross section of the static or growing root.

\medskip

Researchers have done considerable work on modelling auxin-cytokinin crosstalk \cite{muraro2013, mahonen2014, band2014, dimambro2017, salvi2020, mellor2020} and some research into auxin-ethylene crosstalk \cite{moore2015}. However, mathematical models of auxin-BR crosstalk have been limited to a horizontal cross section of a static root \cite{ibanes2009}. Additionally, only a small subset of existing models incorporate root growth, which produces important feedback loops with the hormone gradient \cite{rutten2022}. It is from these observations that the research questions for this paper are identified:
\begin{itemize}
	\item How does BR influence auxin distributions in static and growing roots? 
	\item Is BR-auxin crosstalk sufficient to replicate experimentally identified growth rates across the root?
\end{itemize}


This paper proceeds with a brief overview of biological information about \emph{Arabidopsis} roots, followed by an exploration of prior modelling methods and results. Then, a model incorporating auxin, cytokinin, ethylene, and BR will be developed and tested against previous results.

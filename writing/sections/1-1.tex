\definecolor{brown1}{RGB}{251,235,212}
\definecolor{brown2}{RGB}{208,185,152}
\definecolor{brown3}{RGB}{173,146,107}
\definecolor{brown4}{RGB}{155,125,85}
\definecolor{brown5}{RGB}{138,112,74}

\section{Root Morphology}

\subsection{Structure and Regions}

The interactions between auxin and other hormones partitions the root into four zones with distinct growth activities \cite{verbelen2006}.

\begin{center}
\begin{tikzpicture}

	\node[draw,align=center,minimum width = 4cm,minimum height = 2cm, fill=brown5] at (0,5) {\small Mature Zone (MZ)};
\node[draw,align=center,minimum width = 4cm, minimum height = 2cm, fill=brown4] at (0,3) {\small Elongation Zone (EZ)};
\node[draw,align=center,minimum width = 4cm, minimum height = 1cm, fill=brown3] at (0,1.5) {\small Transition Zone (TZ)};
\node[draw,align=center,minimum width = 4cm, minimum height = 1cm, fill=brown2] at (0,0.5) {\small Division Zone (DZ)};

\draw[gray, thick, name path = A] (-2, 0) -- (2, 0);
\draw[gray, thick, name path = B] (-2, 0) .. controls (-2, -4) and (2, -4) .. (2, 0);

\fill [brown1, intersection segments={of=A and B, sequence={L2--R2}}];
\draw[gray] (-2, 0) .. controls (-1, -1) and (1, -1) .. (2, 0);


\node[align=center, minimum height = 1cm] (A) at (4, 6) {Surface};
\node[align=center, minimum height = 1cm] (B) at (4, -2) {Root Tip};
\draw [->, line width = 0.5mm] (A) edge (B);

\end{tikzpicture}
\end{center}

\medskip

Vascular, border, and epidermal cell columns make up the primary root shaft. The flow of the various hormones between these columns will be explored further in later sections. Underneath the columns of cells is the root cap, which protects the root as it grows \cite{kumpf2015}. In order to preserve this function, root cap cells are continuously created and destroyed \cite{kumpf2015}, and thus most models of the growing root assume the root tip is static \cite{mahonen2014, salvi2020}. At the top of the root cap is the important quiescent centre (QC), which aids in regeneration of the root cap and promotes division of the cells around it \cite{matosevich2021}.  

\medskip

The division zone spans up to $\um{200}$  away from the root cap junction. In this region, cells grow from $\um{4.5}$ up to $\um{9}$ over the course of approximately $18$ hours. On average, the division zone contains around $21$ cells per file, and all cells are mitotic. The transition zone ranges from the top of the division zone to about $\um{520}$ above the root cap junction. Cells in this zone grow to $\um{30}$ over $10$ hours. In the transition zone, vascular cells are not mitotic, but border and epidermal cells are. Rapid cell elongation occurs in the next region, up to $\um{900}$ above the root cap junction. Cells in the elongation zone grow to $\um{135}$ in $4$ hours, before entering the mature zone at their maximum size \cite{verbelen2006}.


\subsection{Auxin Influx and Efflux}

The PIN-FORMED (PIN) family of auxin transporters is responsible for mediating auxin efflux within the growing root \cite{krecek2009}. The polarized distribution of PIN proteins within each cell, including PIN1, PIN2, PIN3, PIN4, and PIN7, is instrumental in producing a stable auxin gradient \cite{yang2020}. The amount of PIN transporters on a cell membrane is mediated by intracellular recycling \cite{paciorek2005, ambrose2013}. Auxin has been shown to inhibit this process \cite{paciorek2005}, which increases the amount of PIN on the membrane. Conversely, BR promotes PIN recycling through direct transcriptional regulation \cite{hacham2012} and indirectly through CLASP \cite{ambrose2013}.

\medskip

Auxin influx is determined by the nonpolar AUX/LAX family of transporters \cite{swarup2012} as well as passive diffusion through the cell membrane. Members of the AUX/LAX transporter family perform similar functions but are located in different regions of the cell. For instance, AUX1 is concentrated in the border tissue \cite{swarup2001}, while LAX2 and LAX3 are located in the vascular tissue and columella \cite{swarup2012}. Together, the PIN efflux transporters nad AUX/LAX influx transporters produce a distinct auxin gradient within the cell \cite{band2014}. In healthy roots, this gradient is characterized by an auxin maximum at the QC, gradually decreasing auxin levels up through the vascular tissue, and low auxin levels in the border tissue \cite{grieneisen2007}.

\subsection{PLETHORA}

When exposed to high concentrations of auxin for a long period of time, \emph{Arabidopsis} roots experience an increase in PLETHORA transcription factors (PLT) \cite{mahonen2014}. PLT is known to upregulate the transcription of the YUCCA3 gene, which increases auxin biosynthesis. Additionally, PLT modulates cell elongation and differentiation on a much longer time scale than auxin. This allows the root to respond quickly to environmental stimuli via auxin redistribution, while maintaining stable zonation due to transcriptional regulation of PLT \cite{mahonen2014}. 

\subsection{Cytokinin}

Cytokinin is an important phytohormone with many important functions in the gorowing root. It has been shown to inhibit auxin flow through the vascular and columella tissue by restricting the expression of PIN1, PIN3, and PIN7 \cite{Ioio2008} via the SHY2 protein and ARR1/ARR12 signalling pathways. Additionally, cytokinin promotes the auxin inhibiting GH3 protein \cite{dimambro2017} and upregulates cell differentiation \cite{Ioio2008}. 

\subsection{Brassinosteroid, CLASP, and Microtubules}


Brassinosteroid (BR) is an essential part of the hormone network in \emph{Arabidopsis} roots, and has shown to promote both longitudinal and radial growth in a spatiotemporal manner \cite{ackerman-lavert2020}. Unlike auxin, BR is not transported over long distances through the vascular tissue, but can diffuse locally \cite{vukasinovic2021}. Each cell synthesizes its own BR, based on a gradient of signalling enzymes that reaches a maximum in the elongation zone \cite{vukasinovic2021}.  In the vascular tissue, BR has been shown to interact antagonististically with auxin in order to maintain the division and elongation zones \cite{chaiwanon2015}. Notably, BR is also known to promote auxin biosynthesis in the epidermis, which unveils a context-specific relationship between BR and auxin \cite{vragovic2015}. BR also indirectly influences auxin through transcriptional regulations of the PIN2 and PIN4 transporters \cite{hacham2012}.

\medskip

Microtubules (MTs) are bundles of cellulose that form on the outside of cells. Depending on their orientation, MTs can restrict or induce growth in the cell \cite{ambrose2011}. Transverse MT arrays promote due to their limited resistance along the axis of growth. On the other hand, transfacial bundles created by the protein CLASP will restrict growth in all directions. BR has been shown to inhibit CLASP and thus promote growth \cite{ruan2018}. However, CLASP also upregulates the BR receptor BRI1, which creates a stable positive-negative feedback loop \cite{ruan2018}. Additionally, CLASP has been shown to promote the recycling of PIN2 \cite{ambrose2013}, but further research is needed to determine the complete crosstalk between CLASP and auxin.

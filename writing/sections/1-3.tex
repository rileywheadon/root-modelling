\section{Growing Root Models}

Limited studies have attempted to model hormone gradients within growing roots \cite{grieneisen2007, muraro2013, mahonen2014, salvi2020} and these studies have applied relatively simplistic descriptions of cell and tissue growth \cite{rutten2022}. Other studies have built highly accurate models of root growth using vertex-element models \cite{fozard2013, fozard2016}, but have not yet incorporated auxin dynamics. All models of root growth face the problem of \emph{symplastic growth}, which refers to the idea that cells are unable to move independently of one another because they share a cell wall with their neighbours \cite{ivanov2002}. This implies that any two cells equidistant from the root tip must be growing at precisely the same rate, at all times \cite{ivanov2002}. Due to the fact that the distributions of auxin and other hormones are spatially regulated in both the transverse and longitudinal direction, defining a growth function that is constant along the horizontal axis is challenging. Grieneisen et al. \cite{grieneisen2007} tackle this problem using a cellular potts model that restricts growth to the longitudinal direction to prevent cell sliding.  The work of M\"{a}ho\"{n}en et al. \cite{mahonen2014}, later built on by Salvi et al. \cite{salvi2020}, instead determine growth by taking an average of the growth factors in all vascular cells and applying it to the entire row. Cells "grow" via the addition of a new row of simulation points to the lattice. Models of growing roots find strong evidence that root growth affects hormone dynamics \cite{rutten2022}, making solving the problem of modelling symplastic growth an important issue for further study.




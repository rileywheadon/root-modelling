The CLASP protein has been shown to influence cell growth by changing the arrangement of cellulose microfibrils via the orientation of microtubule arrays (MTs) on the cell membrane.
Here we present a mathematical model of this phenomenon which we verify against \emph{in vivo} observations of \emph{A. thaliana} roots. 
Our model uses data from the brassinosteroid (BR) signalling pathway and predicts its downstream effects on CLASP, MTs, and ultimately root growth. 
Additionally, the model accurately explains the behaviour of various mutant roots in which elements of the aforementioned signalling network are edited or removed. 
We find that BR signalling alone is unable to predict observed rates of cell growth in the root and conclude that CLASP is essential for root zonation.


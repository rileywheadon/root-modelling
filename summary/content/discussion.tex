
\subsection{The Single Cell Model}

Many of the parameters presented were prescribed or assigned prior to fitting the model. Some of these parameters, such as $K_{d}$ (the BL dissociation constant) and $R_{T}$ (the total concentration of BRI1 receptors) were based on previous results in the literature. However, other parameters such as $n$ (the BL moving average period) and $b$ (the CPD/ROT3 bias) were prescribed  with minimal justification. This was necessary due to the fact that the data available was only able to support a small number of parameters without running the risk of overfitting. In light of new data, it may be possible to get estimates and confidence intervals for the assigned parameters.

\medskip

Further experiments are necessary to more accurately fit the model. In particular, quantitative measurements of extracellular BR concentrations, CLASP protein levels, and the BES1 transcription factor would lead to significant improvements in model efficacy. The work in this paper also presents interesting oppourtunities for further modelling. Adding a model of cell division to the system of ODEs presented here would make it possible to describe an entire column of trichoblast or atrichoblast cells. This cell column model could be integrated into a two-dimensional cross section model (\cite{grieneisen2007}, \cite{dimambro2017}, \cite{salvi2020}), which often omit the effects of BR and CLASP on growth and division. This would make it possible to study the crosstalk between BR and auxin (\cite{chaiwanon2015}, \cite{vragovic2015}) \emph{in silico} to help us better understand how these crucial plant hormones interact with one another.

\subsection{Data Limitations}

% The descriptions and results from the additional models are contained in:
% - growth-model.tex
% - clasp1-model.tex
% - brin-clasp-model.tex

We developed models of cell growth using time-dependent ODEs but soon realized that such models could not be extended to the CLASP-1 and BRIN-CLASP mutants. The reason for this is that the data from \cite{goh2023} was crucial for converting position-dependent observations into time-dependent observations. However, this data was taken from wild type roots, and the position vs. time function we approximated is likely different for the mutants. In order to validate the time-dependent models presented in the previous section with experimental data, we would need multiple images of the CLASP-1 and BRIN-CLASP root tips over a period of time. Unfourtunately, this data is not currently available.



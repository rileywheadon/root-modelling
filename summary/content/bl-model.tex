\subsection{Brassinoslide Component}

\cite{vukasinovic2021} provide a detailed biosynthetic pathway for BL by outlining how campestrol undergoes multiple chemical reactions to eventually form BL. The biosynthetic enzymes DWF4, CPD, DET3, ROT3, BR6OX1, and BR6OX2 catalyze these reactions. Mutants made to be deficient in each of the aforementioned enzymes exhibited stunted growth, which shows that these enzymes are crucial for effective BL synthesis. The biosynthetic enzymes CPD and ROT3 were observed by \cite{vukasinovic2021} using fluorescence imaging in the vascular cell columns of a single root. They found that both CPD and ROT3 remain relatively constant in the meristem (0-200\um) before increasing in the transition zone (200-400\um) and reaching a maximum in the early elongation zone (400-700\um). After 700\um, the CPD and ROT3 levels began to decrease.

\medskip

Since BL is the most biosynthetically active brassinosteroid, we make the assumption that it is the only brassinosteroid present in the root. Additionally, we assume that the fluorescence intensity of CPD and ROT3 measured by (\cite{vukasinovic2021}) are equivalent to the concentration of BL ligand up to some scalar multiple. To determine this scalar, we use an estimate from \cite{vanesse2012} that gives the extracellular concentration of BL ligand to be at most 1\nm in wild-type roots. This gives us the following formula for the extracellular BL concentration [BL] in \nm:

\begin{equation}
    \label{bl}
[\text{BL}] = b \cdot \frac{\text{CPD}}{\text{max}(\text{CPD})} + (1 - b) \cdot \frac{\text{ROT3}}{\text{max}(\text{ROT3})}
\end{equation}

\medskip

In \eqref{bl}, the parameter $b \in [0, 1]$ controls the bias towards CPD in order to account for the fact that the exact details of the BL pathway are omitted from our model. Using the data from \cite{vukasinovic2021} along with \eqref{bl}, we plotted the BL concentration against position for different levels of bias (see Supplementary Figure \ref{sfig:bl-bias}). Since the functions behaved qualitatively the same, we assigned $b = 0.5$. Additionally, BR biosynthetic enzymes have been shown to move short distances in the root (\cite{vukasinovic2021}), so we take an $n\,\um$ moving average of the BL concentration function to account for diffusion. Shown in Supplementary Figure \ref{sfig:bl-average} is a plot of the BL concentration function for different values of $n$. We settled on $n = 50\um$ as a reaosnable estimate of this diffusive effect.

